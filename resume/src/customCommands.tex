\usepackage{hyperref}
\usepackage{titlesec}
\usepackage{xparse}
\usepackage{fontawesome5}
\usepackage{xstring}

\newcommand{\email}[1]{\faEnvelope \href{mailto:#1}{\ #1}}
\newcommand{\github}[1]{\faGithub \href{#1}{\ \StrSubstitute{#1}{https://}{}{}}}
\newcommand{\name}[1]{\textbf{#1}}
\newcommand{\homepage}[1]{\faHome \href{#1}{\ \StrSubstitute{#1}{https://}{}{}}}
\newcommand{\blog}[1]{\faBlog \href{#1}{\ \StrSubstitute{#1}{https://}{}{}}}

% Set title format for sections
\titleformat{\section}
{\Large\bfseries}
{}
{0em}
{\vspace{-.8em}}[]

% A newcommand named cventry is defined. It takes 5 arguments.
% The first argument is the title of the entry,
% the second is the subtitle,
% the third is the location,
% the fourth is the date,
% and the fifth is the description.

\def\customTextWidth{\textwidth}

\newcommand{\cventry}[5]{
  \noindent
  \vspace{-2pt}
  \begin{tabular*}{\customTextWidth}[t]{l@{\extracolsep{\fill}}r}
    \textbf{#1} & {\small #2}\vspace{.2pt}\\ % top row of resume entry
    \textit{#3} & {\small #4} % second row of resume entry
  \end{tabular*}
  \vspace{-4pt}
  #5
}

% The first argument is the title of the project,
% The second is link
% The third is the description of the project.
\newcommand{\onelineproject}[3]{
  % \vspace{.8em}
  % \noindent
  \textbf{\href{#2}{#1}} \vspace{.2pt} #3
}

\newcommand{\onelineprojectNoLink}[2]{
  % \vspace{.8em}
  % \noindent
  \textbf{#1}\vspace{.2pt} #2
}

% cvitem is a newenvironment wrapping around the itemize environment.
\newenvironment{cvitem}{
  \begin{itemize}[noitemsep]
}{\end{itemize}}
